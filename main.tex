\documentclass{article}
\usepackage[margin=1in]{geometry}
\usepackage[utf8]{inputenc}
\usepackage{amsmath}
\usepackage[parfill]{parskip} % new line between paragraphs, no indentation
\usepackage{enumerate} % change the symbol of enumerate
\usepackage{xcolor}
\usepackage{soul} % highlight texts
\usepackage{natbib}
\usepackage{graphicx}
\usepackage{hyperref}

% Header
\usepackage{fancyhdr}
\pagestyle{fancy}
\fancyhf{}%Clear all heads and foots
\setlength{\headheight}{35pt} %Eliminate the warning of "headheight is too samll"
\rhead{Kriging Interpolation\\Jianzhao Bi\\\today}
\cfoot{\thepage}


\begin{document}

\section{Ordinary Kriging}

\subsection{General Equation}
\begin{equation}
\label{emu:ok}
    \hat{z_0}=\sum_{i=0}^{n}\lambda_iz_i
\end{equation}
where $\hat{z_0}$ is the estimation of $z$ in $(x_0, y_0)$. $\lambda_i$ is the weight. It should be noted that this equation is established in each unknown point, that is, $\lambda_i$ are different for each unknown point.

The weights can be solved by considering two constraints:
\begin{equation}
\label{emu:res2}
    \min\limits_{\lambda_i}Var(\hat{z_0}-z_0)
\end{equation}
and 
\begin{equation}
\label{emu:equalexp}
    E(\hat{z_0}-z_0)=0
\end{equation}

\subsection{Hypotheses and Constraints}
The hypotheses for the Ordinary Kriging include the equal expectation and variance in any point $(x,y)$, \textit{i.e.,}
\begin{align}
    E[z(x,y)]=E[z]=c\\
    Var[z(x,y)]=\sigma^2
\end{align}

So, the constraint (\ref{emu:equalexp}) becomes
\begin{equation}
\label{emu:res1}
    \sum_{i=0}^n\lambda_i=1
\end{equation}

\subsection{Semivariance Function}
The semivariance between points $i$ and $j$ is defined as the half of the variance between the two points
\begin{equation}
    r_{ij}=\sigma^2-C_{ij}=\frac{1}{2}E[(z_i-z_j)^2]
\end{equation}

The semivariance function is used to estimate the semivariance between a known point and an unknown point. We assume it is correlated to the distance between the points. Thus, this relationship can be fitted by a function of the distance between two known points ($h$) and their known semivariance $r$. 
\begin{equation}
\label{emu:semi}
    r=r(h)
\end{equation}

\subsection{Interpolation Steps}
\begin{enumerate}[1)]
    \item Calculating the semivariances between all pairs of known points $z_i$ and $z_j$ ($r_{ij}$) and fitting the semivariance function (\ref{emu:semi}) ($r_{ij}=r(h_{ij})$)
    \item Calculating the semivariances between the unknown point $z_0$ and known points $z_i$ ($r_{0i}$) using the semivariance function (\ref{emu:semi})
    \item Using the constraints (\ref{emu:res1}), (\ref{emu:res2}) and estimated semivariances $r_{0i}$, calculating the weights $\lambda_i$ for this unknown point
    \item Using Equation (\ref{emu:ok}) to estimate the value of the unknown point
\end{enumerate}

\section{Kriging with External Drift}

When the mean is not equal in the domain, other Kriging methods should be used:
\begin{itemize}
    \item “Universal kriging” (UK) is as a special case of Kriging with changing mean where \textbf{the trend is modelled as a function of coordinates}. 
    \item If, instead of using monomials of the coordinates in the UK equations, \textbf{the drift is defined externally through some auxiliary variables}, the term “Kriging with external drift” (KED) or external trend is used.
\end{itemize}

\subsection{General Equation}
In the case of KED, predictions at new locations are made by:
\begin{equation}
    \hat{z}_{KED}(0)=\sum_{i=0}^{n}\lambda_i(0)^{KED}z(i)
\end{equation}

\subsection{Hypotheses and Constraints}
\begin{equation}
    \sum_{i=1}^n\lambda_i^{KED}(0)\cdot q(i)=q(0)
\end{equation}
where $q(i)$ is the value of the auxiliary variable (aka. predictor) (\textit{e.g.,} DEM) in the location $i$.

\section{Universal Kriging}
\subsection{General Equation}
New points are estimated by 
\begin{equation}
    \hat{Z}_0=\sum_{j=1}^n\lambda_jZ(\mathbf{x}_j)
\end{equation}
where $\mathbf{x}_j$ is the coordinate vector of the point $j$. For example, for two-dimensional coordinates, $\mathbf{x}=(x_1, x_2)$.

The assumption of the mean is 
\begin{equation}
    \begin{split}
        E[Z(\mathbf{x})]=m(\mathbf{x}) \\
        m(\mathbf{x})=\sum_{l=0}^La_lf_l(\mathbf{x})
    \end{split}
\end{equation}
with known functions $f_l(\mathbf{x})$ and unknwon $a_l$'s. $L$ is the number of the functions $f_l(\mathbf{x})$. For example, for a polynomial trend, $f_0(\mathbf{x})=1$, $f_1(\mathbf{x})=x_1$, $f_2(\mathbf{x})=x_2$, $f_3(\mathbf{x})=x_1x_2$, \textit{etc.}. 

\subsection{Hypotheses and Constraints}
For any combination of $a_l$'s,
\begin{equation}
    f_l(\mathbf{x}_0)=\sum_{j=1}^n\lambda_jf_l(\mathbf{x}_j)
\end{equation}
These are $L+1$ constraints, corresponding to $L+1$ $a_l$'s. 

\section*{References}
\begin{itemize}
    \item Kriging Interpolation (Chinese): \url{https://xg1990.com/blog/archives/222}
    \item Universal Kriging: \url{http://infohost.nmt.edu/~olegm/586/HYD11-16.pdf}
    \item Hengl, T., Heuvelink, G. B. M., \& Stein, A. (2003). Comparison of kriging with external drift and regression-kriging. Technical note, ITC.
\end{itemize}

\end{document}
